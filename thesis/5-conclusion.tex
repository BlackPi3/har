\chapter{Conclusion}
\label{chap:conclusion}

In this study, we presented a novel adversarial learning approach to Human Activity Recognition (HAR) that enhances activity classification by generating synthetic accelerometer data from 3D skeleton poses. While the baseline method follows an end-to-end pipeline—where a pose-to-sensor network and an activity classifier are trained concurrently—our key contribution lies in integrating adversarial learning on top of this pipeline. By introducing a discriminator that distinguishes between real and synthetic sensor data, the adversarial framework forces the generated data to better match the distribution of real sensor data, leading to superior classification performance.

The adversarial learning method demonstrated significant improvements in model accuracy and F1 scores compared to the end-to-end pipeline without the discriminator. This highlights the advantage of adversarial learning in refining the quality of synthetic data and, consequently, the robustness of the activity classifier. While the end-to-end pipeline offers a solid foundation by training the pose-to-sensor network and classifier simultaneously, the inclusion of adversarial learning ensures a higher degree of realism in the synthetic data, which translates into better generalization for real-world HAR tasks.

However, the adversarial approach currently has limitations. It relies on a secondary dataset that shares activity classes with the primary dataset, and as a result, the performance gains are mostly observed on a subset of the primary dataset where such overlap exists. Expanding this framework to handle more diverse activity classes and datasets remains an important area for future exploration.

One promising direction for future research is the potential to leverage annotated “in the wild” videos to augment the target dataset. These publicly available videos, containing naturally occurring human activities, offer a rich source of training data. Incorporating such data into the adversarial framework could further improve the model’s ability to generalize to real-world scenarios, making HAR systems more robust and applicable to everyday tasks.

The results of this study affirm the potential of adversarial learning in addressing the challenges posed by limited labeled datasets in HAR. By optimizing both sensor data generation and classification in a competitive framework, our approach improves the realism of the synthetic data and enhances overall model performance.

Looking ahead, there are several avenues for extending this work. Exploring alternative training strategies, such as multi-task learning, could optimize related tasks simultaneously and lead to further improvements in HAR systems. Additionally, enhancing the scalability of the adversarial framework to handle a wider range of activities and more diverse datasets could significantly increase its applicability in practical scenarios.